\documentclass[letterpaper]{article}
\newcounter{majornumber}
\newcounter{minornumber}
\setcounter{majornumber}{1}
\setcounter{minornumber}{1}

\usepackage[utf8]{inputenc}

\usepackage[margin=1in]{geometry}

\usepackage[table,usenames]{xcolor}
\definecolor{linkColor}{RGB}{0, 0, 255}
\usepackage[pdfauthor={Paul\ M.\ Heider},%
            pdftitle={Cancer Trial Eligibility Annotation Schema},%
            pagebackref=true,%
            linktoc=section,%
            linkcolor=linkColor,%
            citecolor=linkColor,%
            urlcolor=linkColor,%
            colorlinks=true,%
            unicode,psdextra]{hyperref}

\usepackage{soul}
\usepackage{tikz}
\usetikzlibrary{calc,positioning}

\usepackage{gb4e}
\noautomath

\usepackage{fancyhdr}
\rfoot{(v. \themajornumber.\theminornumber)}
\title{Cancer Trial Eligibility Annotation Schema\\(v. \themajornumber.\theminornumber)}
\author{Translational Biomedical Informatics Center at MUSC}

\tikzstyle{information text}=[rounded corners,fill=black!10,inner sep=1ex]
\tikzstyle{attrbox}=[align=center,shape=rectangle,draw,inner sep=5pt, rounded corners=4pt, thick,fill=black!10]

\newcommand{\prefixDeep}[3]{%
  #1 %
  \tikz[baseline={([yshift={-\ht\strutbox}]a\x.north)},outer sep=0pt,inner sep=0pt]{%
    \node[align=center] (a\x) {%
      \strut \hl{#2}};%
  } %
  #3
}
\newcommand{\suffixDeep}{%
  \tikz[overlay]{ \draw[<->,thick] (b\x.west) -| (a\x.south); }%
  \pgfmathparse{int(\x+1)}%
  \xdef\x{\pgfmathresult}%
}
\newcommand{\oneDeep}[3]{%
  \begin{flushright}
  \tikz{
    \node[attrbox,align=right] (b\x) {%
      #2:  #3%
    };%
  }
  \end{flushright}
  \suffixDeep
}
\newcommand{\twoDeep}[4]{%
  \begin{flushright}
  \tikz{
    \node[attrbox,align=right] (b\x) {%
      #1:  #2\\%
      #3:  #4%
    };%
  }
  \end{flushright}
  \suffixDeep
}
\newcommand{\threeDeep}[6]{%
  \begin{flushright}
  \tikz{
    \node[attrbox,align=right] (b\x) {%
      #1:  #2\\%
      #3:  #4\\%
      #5:  #6%
    };%
  }
  \end{flushright}
  \suffixDeep
}
\newcommand{\fiveDeep}[9]{%
  \begin{flushright}
  \tikz{
    \node[attrbox,align=right] (b\x) {%
      #1:  #2\\%
      #3:  #4\\%
      #5:  #6\\%
      #7:  #8\\%
      Historical:  #9%
    };%
  }
  \end{flushright}
  \suffixDeep
}

\newcommand{\staging}[4]{
  \threeDeep{#1}{#2}{Timing}{#3}{Historical}{#4}%
}
\newcommand{\tumor}[3]{\staging{Tumor}{#1}{#2}{#3}}
\newcommand{\lymphNodes}[3]{\staging{Nodes}{#1}{#2}{#3}}
\newcommand{\metastases}[3]{\staging{Metastases}{#1}{#2}{#3}}

\newcommand{\menopause}[2]{\twoDeep{Postmenopausal Status}{#1}{Historical}{#2}}
\newcommand{\ecog}[2]{\twoDeep{ECOG}{#1}{Historical}{#2}}

\newcommand{\erStatus}[5]{\fiveDeep{ER Status}{#1}{ER Positivity}{#2}{Allred Status}{#3}{Allred Score}{#4}{#5}}
\newcommand{\prStatus}[5]{\fiveDeep{PR Status}{#1}{PR Positivity}{#2}{Allred Status}{#3}{Allred Score}{#4}{#5}}
\newcommand{\herStatus}[3]{\threeDeep{HER2 Status}{#1}{HER2 Positivity}{#2}{Historical}{#3}}

\begin{document}
\tikzstyle{every picture}+=[remember picture]
\def\x{0}

\maketitle

\paragraph{Introduction:}
This document explains and provides guidelines for annotation of cancer trial eligibility criteria that are relevant for the Clinical Trial Eligibility project. 
It also serves to maintain consistency across all annotators. 
The natural language processing tool will learn from the expert annotations and then automatically extract this information from electronic health records of various types.
Annotations will be done to help localize spans of text that contain evidence for a particular criterion and the conclusions to be drawn from said evidence.
The annotation will be done using an annotation tool available online: WebAnno. 
%%Each report will be independently annotated by two experts, and a third expert will adjudicate any disagreement.

\paragraph{Conventions applied to classes and slots below:}
In the provided examples, highlighted text provides examples of what to annotate.
Annotate the full span of text containing the measurement or value of interest and the label associated with that value.
Some of these measurements are reports of historical facts.
For those annotations, you should select the \textsl{Historical} checkbox.
The value that you set an attribute to should reflect the status at the time the measurement is taken, even if that value differs from the current status.
Some of the measurements will be undefined, unreported, or ambiguous.
For those annotations, you should leave the value blank (if numerical) or choose \textsl{Indeterminant} from the drop-down.
Beside the highlighted span is an information box showing the attribute(s) to set for the given span.
Labels and measurement values may appear in other formats not documented below but nontheless should be annotated.

\begin{enumerate}
\item
  Postmenopausal Status

  \textbf{Values:}  \textsl{True}, \textsl{False}, \textsl{Indeterminant}
  
  Menopausal status is sometimes conveyed as a simple categorical attribute.
  Menopausal status can be given '\textsl{at diagnosis}' in a report far removed from the original diagnosis.
  These status values should be treated as \textsl{Historical}.
  In these historical cases, the value that you set \textsl{Postmenopausal Status} to should reflect the status at that point in time, regardless of the current status.
  When the last menstrual period (LMP) is given in date format, cancer trial eligibility requires that the last menstrual period be more than 2 years ago.

  Examples:
  \begin{exe}
  \item
    \prefixDeep{}{MENOPAUSE STATUS: Post menopausal at diagnosis}{}
    \menopause{True}{false}
    
  \item
    \prefixDeep{}{Pre-menopausal at diagnosis}{}
    \menopause{False}{true}
    
  \item
    \prefixDeep{}{now post-menopausal}{}
    \menopause{True}{false}
    
  \item
    \prefixDeep{}{LMP 01/01/1992}{}
    \menopause{True}{false}

  \item
    \prefixDeep{Report Date:  11/27/2017 {\dots}}{LMP 11/01/2017}{}
    \menopause{False}{false}
    
  \end{exe}

\item
  ECOG Scale of Performance Status\footnote{\url{http://ecog-acrin.org/resources/ecog-performance-status}}
  
  \textbf{Values:}  \textsl{0}--\textsl{5}
  
  Examples:
  \begin{exe}
  \item
    \prefixDeep{}{ECOG status 0}{}
    \ecog{0}{false}
    
  \item
    \prefixDeep{Currently her}{ECOG is 3}{due in part to \dots}
    \ecog{3}{false}
    
  \end{exe}
  
\item
  Hormone Receptor Status

  Each of the three hormone receptors (\textsl{ER}, \textsl{PR}, and \textsl{HER2}) have four attributes of interest:
  \begin{enumerate}
  \item
    ER, PR, and HER2 Status

    \textbf{Values:}  \textsl{Positive}, \textsl{Negative}, \textsl{Indeterminant}

    The hormone receptor status is sometimes described in categorical terms.
    If an explicit categorical \textsl{Positive} or \textsl{Negative} is not given, then you should consider the attribute \textsl{Indeterminant}.
    
  \item
    ER, PR, and HER2 Positivity

    \textbf{ER and PR Values:}  \textsl{0}--\textsl{100} \\
    \textbf{HER2 Values:}  \textsl{0}--\textsl{3}

    The hormone receptor positivity is a percentage (ranging from zero to 100 for ER and PR and ranging from zero to three for HER2).
    When an explicity number is not provided in the text span, leave this attribute empty.
    
  \item
    Allred Status

    \textbf{Values:}  \textsl{Positive}, \textsl{Negative}, \textsl{Indeterminant}

    The Allred status is sometimes described in categorical terms.
    If an explicit categorical \textsl{Positive} or \textsl{Negative} is not given, then you should consider the attribute \textsl{Indeterminant}.
    
  \item
    Allred Score

    \textbf{Values:}  \textsl{0}--\textsl{8}
    
    The Allred score ranges from zero to eight.
    Leave this attribute empty if only the Allred intensity or proportion score is provided and not the composit Allred score.
    
  \end{enumerate}

  Examples:
  \begin{exe}
  \item
    \prefixDeep{}{ER positive(92\%,3+)}{, PR positive(96\%, 2+), HER unamplified(ratio 1.01 FISH)}
    \erStatus{Positive}{92}{(blank)}{(blank)}{false}

  \item
    \prefixDeep{ER positive(92\%,3+),}{PR positive(96\%, 2+)}{, HER unamplified(ratio 1.01 FISH)}
    \prStatus{Positive}{96}{(blank)}{(blank)}{false}

  \item
    \prefixDeep{ER positive(92\%,3+), PR positive(96\%, 2+),}{HER unamplified}{(ratio 1.01 FISH)}
    \herStatus{Negative}{(blank)}{false}

  \item
    \prefixDeep{}{HER-2/neu amplified}{}
    \herStatus{Positive}{(blank)}{false}
    
  \item
    \prefixDeep{}{o HER2/NEU: 3+ (POSITIVE)}{}
    \herStatus{Positive}{3}{false}
    
  \item
    \begin{samepage}
      \prefixDeep{}{ER focal positive (<1\%; negative by Allred)}{}
      \erStatus{Positive}{<1}{Negative}{(blank)}{false}
    \end{samepage}
    
  \end{exe}
  
\item
  Breast Cancer Staging\footnote{\url{https://cancerstaging.org/references-tools/quickreferences/Documents/BreastMedium.pdf}}

  \begin{enumerate}
  \item
    Timing

    \textbf{Values:}  \textsl{Clinical}, \textsl{Pathological}

    The staging values can be determined clineically or pathologically.
    Leave this attribute empty if no explicit timing information is given.
    
  \item
    Primary Tumor (T)

    \textbf{Values:}  \textsl{T0}, \textsl{T1}, \textsl{T2}, \textsl{T3}, \textsl{T4}
    
  \item
    Regional Lymph Nodes (N)

    \textbf{Values:}  \textsl{N0}, \textsl{N1}, \textsl{N2}, \textsl{N3}

  \item
    Distant Metastases (M)

    \textbf{Values:}  \textsl{M0}, \textsl{M1}
    
  \end{enumerate}

  Examples:
  \begin{exe}
   \item
     \prefixDeep{Invasive Ductal carcinoma with associated DCIS,}{pT2}{pN1mi (sn)}
     \tumor{T2}{Pathological}{false}
    
   \item
     \prefixDeep{Invasive Ductal carcinoma with associated DCIS, pT2}{pN1mi}{(sn)}
     \lymphNodes{N1}{Pathological}{false}

  \item
    \prefixDeep{cT0 cN1}{cM0}{, Stage IIA}
    \metastases{M0}{Clinical}{false}
    
  \end{exe}
  
\end{enumerate}

\end{document}
